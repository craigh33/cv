% YAAC Another Awesome CV LaTeX Template
%
% This template has been downloaded from:
% https://github.com/darwiin/yaac-another-awesome-cv
%
% Author:
% Christophe Roger
%
% Template license:
% CC BY-SA 4.0 (https://creativecommons.org/licenses/by-sa/4.0/)
%Section: Work Experience at the top
\sectionTitle{Employment History}{\faSuitcase}
%\renewcommand{\labelitemi}{$\bullet$}
\begin{experiences}
  \experience
    {Present}   {Software Engineer}{N-able}{United Kingdom}
    {Oct 2022} {
                      At N-able, I was a member of multiple teams spanning multiple projects, from N-able's Cloud management solution to its latest Agentic AI offering. This position has boosted my skills and given me many opportunities to grow as an engineer, working with modern technology stacks. I progressed from Software Engineer I to Software Engineer II over time, giving me the experience required to pursue more senior roles.
                      \begin{itemize}
                        \item Agentic AI Development - I was a key member of the Agentic AI team, helping to assemble and build agents compliant with A2A in Golang. This new producnt called N-zo, was aimed at giving our customers another tool to manage their devices in the UEM space. I was involved in developing the agents and our multi-agent architecture, as well as our agent tooling via MCP servers, giving me a good understanding of best practices in prompt engineering and tool design. Because of my involvement in the Agentic AI team, I was given the opportunity to expand this knowledge with a trip to AWS re:Invent, where I learned about how the industry is progressing toward an Agentic AI future.
                        \item Microservices Development - I was involved in the development of a new microservice architecture for N-able's Cloud management solution. This involved working in a new tech stack, which included SAGA patterns, document databases, and orchestration via the KNative Kubernetes stack. This platform was used to replace an Azure User Resource Synchronisation service that was used to manage the lifecycle of user resources in Azure.
                        \item API Development - Development of APIs is a core skill for a Full Stack Engineer at N-able, and I have implemented a variety of different APIs, including REST, GraphQL, and gRPC, all in Go. This has improved my skills in API design and implementation and has given me a solid foundation in best practices such as API versioning and documentation.
                        \item Cloud Technology - Due to my involvement in the Cloud management solution, I was exposed to both AWS and Azure technologies and best practices. This has not only given me strong fundamentals in cloud architecture and best practices, but also helped me to develop my skills in Terraform and deployment to both AWS and Azure using Jenkins.
                        \item User Experience - As a Full Stack Engineer, I have been involved in the development of several Angular front ends. I was a key contributor to N-able's new N-zo UI and was given the remit to create new components outside of the shared N-able UI library. This has given me a strong grounding in Angular and the customization of DevExtreme components.
                        \item Mentorship - As a mid-level engineer, I have helped mentor and support more junior members of the team. This has involved pair programming, code reviews, and providing guidance on N-able's technologies and best practices. 
                      \end{itemize}
                    }
                    {Go, AWS Bedrock, Genkit, MCP, A2A, promptfoo, KNative, Terraform, Kubernetes, Kustomize, Service Fabric, Angular, DevExpress DevExtreme, GraphQL, Apollo, gRPC, REST, C\#, Node.js, Jenkins, Azure, SAGA Patterns, DocumentDB, PostgreSQL, MySQL, Microservices}
  \emptySeparator
  \experience
    {Oct 2022} {Software Engineer}{NCR Corporation, XFS Integration Test Team}{United Kingdom}
    {Aug 2019}    {
                      The XFS IT Team in Dundee is a small team tasked with testing components of NCR's XFS ATM Software product and supporting hardware testing with software utilities. The XFS product is shipped with every NCR ATM in the world and is responsible for the hardware drivers and APIs to control them. As a member of this team, I worked independently on different tasks and projects while coordinating with different development teams and stakeholders.
                      \begin{itemize}
                        \item QA Process Experience - Assisted in the testing of software components to be included in the NCR XFS Software stack, ensuring that critical bugs were not missed before delivering to customers.
                        \item Independent Worker - Designed and developed an automated robotic hardware test application. This consisted of an application on the ATM alongside a Universal Robots collaborative robot. This solution helped address the lack of manual testing during COVID-19 lockdown periods and ensured that transaction targets could still be met.
                        \item Desktop Application Development - I was involved in the development of several new desktop applications for the XFS team. This involved working in a Windows specific environment, using WPF and MVVM patterns. A notable project was the development of a new tool for the hardware testing team, which was used to test not only the hardware but also the software components of the ATM in a controlled environment.
                        \item Experience with CI/CD - Designed and implemented a process to migrate SVN projects to GitHub. This involved creating a CI/CD pipeline where none had previously existed. This has led to quality improvements of both tools and scripts.
                      \end{itemize}
                    }
                    {C\#, WPF, MVVM, Prism, XUnit, NSubstitute, SQLite, Entity Framework Core, Universal Robots URScript, GitHub Actions, SonarQube, Artifactory JFrog, Azure Kubernetes Service, JIRA, ASP.NET SignalR, C++, C++/CLI, Win32 API, SVN, GitHub}
  \emptySeparator
  \experience
    {Aug 2018}     {Summer Intern}{Blue2 Digital}{United Kingdom}
    {June 2018}    {
                      During my internship, I conducted research and development in the field of headless Content Management Systems. This mainly involved migrating data from current Blue2 websites to several well-known headless CMS platforms to discover how difficult it may be to migrate and how doing so could help with the company's workflow.
                      \begin{itemize}
                        \item Web Development Experience - I used a wide range of technologies relevant to the current market, including HTML, CSS, PHP, and JavaScript, and significantly increased my knowledge of object oriented programming.
                        \item Presentation Skills - At the end of my placement, I presented my findings to stakeholders and the company.
                      \end{itemize}
                    }
                    {HTML, CSS, JavaScript, PHP}
\end{experiences}
